% me=0 student solutions (ps file), me=1 - my solutions (sol file),
% me=2 - assignment (hw file)
\def\me{0} \def\num{2} %homework number

\def\due{11:55 pm on Thursday, February 8} %due date

\def\course{CSCI-UA 310-007, Basic Algorithms} %course name, changed only once

% **** INSERT YOUR NAME HERE ****
\def\name{Yan Konichshev}

% **** INSERT YOUR NETID HERE ****
\def\netid{yk2602}

% **** INSERT NETIDs OF YOUR COLLABORATORS HERE ****
\def\collabs{N/A, N/A}


\iffalse

INSTRUCTIONS: replace # by the homework number.  (if this is not
ps#.tex, use the right file name)

Clip out the ********* INSERT HERE ********* bits below and insert
appropriate LaTeX code.  There is a section below for student macros.
It is not recommended to change any other parts of the code.


\fi
%

\documentclass[11pt]{article}


% ==== Packages ====
\usepackage{amsfonts,amsmath}
\usepackage{latexsym}
\usepackage{etoolbox}
\usepackage{totcount}
\usepackage{fullpage}
\usepackage{graphicx}
\usepackage{tikz}
\usepackage{tikz-qtree}
\usepackage[bottom]{footmisc}
\usepackage{enumitem}
\usepackage{hyperref}
\usepackage{mathtools}
\usepackage{listings}
\DeclarePairedDelimiter\ceil{\lceil}{\rceil}
\DeclarePairedDelimiter\floor{\lfloor}{\rfloor}

\setlength{\footskip}{1in} \setlength{\textheight}{8.5in}

\newcommand{\handout}[5]{
	\renewcommand{\thepage}{#1, Page \arabic{page}}
	\noindent
	\begin{center}
		\framebox{ \vbox{ \hbox to 5.78in { {\bf \course} \hfill #2 }
				\vspace{4mm} \hbox to 5.78in { {\Large \hfill #5 \hfill} }
				\vspace{2mm} \hbox to 5.78in { {\it #3 \hfill #4} }
				\ifnum\me=0
				\vspace{2mm} \hbox to 5.78in { {\it Collaborators: \collabs
						\hfill} }
				\fi
		} }
	\end{center}
	\vspace*{4mm}
}

\newcounter{pppp}
\newcounter{pppc}
\newcommand{\prob}{\arabic{pppp}} %problem number
\newcommand{\increase}{\stepcounter{pppp}\stepcounter{pppc}} %problem number

% Arguments: Title, Number of Points
\newcommand{\newproblem}[2]{
	\increase
	\restartlist{subtasks}
	\ifnum\me=0
	\ifnum\prob>0 \newpage \fi
	\setcounter{page}{1}
	\handout{\name{} (\netid), Homework \num, Problem \arabic{pppp}}
	{\today}{Name: \name{} (\netid)}{Due: \due}
	{Solutions to Problem \prob\ of Homework \num\ (#2)}
	\else
	\section*{Problem \num-\prob~(#1) \hfill {#2}}
	\fi
}

\newlist{subtasks}{enumerate}{1}
\setlist[subtasks]{label={(\alph*)},resume}

\newcounter{numpppp}
\loop
\stepcounter{numpppp}
% workaround bug in totcount (means \newtotcounter{points\arabic{pointsct}})
\begingroup%
\edef\tempcounter@@name{points\arabic{numpppp}}%
\expandafter\newtotcounter\expandafter{\tempcounter@@name}%
\edef\tempcounter@@name{ecpoints\arabic{numpppp}}%
\expandafter\newtotcounter\expandafter{\tempcounter@@name}%
\endgroup
\ifnum \value{numpppp}<500 % max number of tasks supported
\repeat

% formating of output
\newcommand{\disppoints}[1]{%
	\texorpdfstring{(#1~\ifnumequal{#1}{1}{point}{points})}{}
}

% adds and displays
\newcommand{\points}[1]{%
	\texorpdfstring{\addtocounter{points\arabic{pppc}}{#1}\disppoints{#1}}{}%
}
\newcommand{\ecpoints}[1]{%
	\texorpdfstring{\addtocounter{ecpoints\arabic{pppc}}{#1}\ec \disppoints{#1}}{}%
}

% total points of current task         
\newcommand{\currentpoints}{% total points of current task   
	\texorpdfstring{\ifnumequal{\totvalue{ecpoints\arabic{pppc}}}{0}%
		{\total{points\arabic{pppc}} points}%
		{\total{points\arabic{pppc}}+\total{ecpoints\arabic{pppc}} points}}{}%
}

\newcommand{\mixedpoints}[2]{%
	\texorpdfstring{%
		\addtocounter{points\arabic{pppc}}{#1}%
		\addtocounter{ecpoints\arabic{pppc}}{#2}%
		(#1 (+#2) points)
	}{}%
}


\def\squarebox#1{\hbox to #1{\hfill\vbox to #1{\vfill}}}
\def\qed{\hspace*{\fill}
	\vbox{\hrule\hbox{\vrule\squarebox{.667em}\vrule}\hrule}}
\newenvironment{solution}{\begin{trivlist}\item[]{\bf Solution:}}
	{\qed \end{trivlist}}
\newenvironment{solsketch}{\begin{trivlist}\item[]{\bf Solution
			Sketch:}} {\qed \end{trivlist}}
\newenvironment{code}{\begin{tabbing}
		12345\=12345\=12345\=12345\=12345\=12345\=12345\=12345\= \kill }
	{\end{tabbing}}


\newcommand{\hint}[1]{({\bf Hint}: {#1})}
% Put more macros here, as needed.
\newcommand{\room}{\medskip\ni}
\newcommand{\brak}[1]{\langle #1 \rangle}
\newcommand{\bit}[1]{\{0,1\}^{#1}}
\newcommand{\zo}{\{0,1\}}
\newcommand{\C}{{\cal C}}

\newcommand{\nin}{\not\in}
\newcommand{\set}[1]{\{#1\}}
\renewcommand{\ni}{\noindent}
\renewcommand{\gets}{\leftarrow}
\renewcommand{\to}{\rightarrow}
\newcommand{\assign}{:=}

\newcommand{\AND}{\wedge}
\newcommand{\OR}{\vee}
\newcommand{\For}{\mbox{\bf for }}
\newcommand{\To}{\mbox{\bf to }}
\newcommand{\DownTo}{\mbox{\bf downto }}
\newcommand{\Do}{\mbox{\bf do }}
\newcommand{\If}{\mbox{\bf if }}
\newcommand{\Then}{\mbox{\bf then }}
\newcommand{\Else}{\mbox{\bf else }}
\newcommand{\Elif}{\mbox{\bf elif }}
\newcommand{\While}{\mbox{\bf while }}
\newcommand{\Repeat}{\mbox{\bf repeat }}
\newcommand{\Until}{\mbox{\bf until }}
\newcommand{\Return}{\mbox{\bf return }}
\newcommand{\Halt}{\mbox{\bf halt }}
\newcommand{\Swap}{\mbox{\bf swap }}
\newcommand{\Ex}[2]{\textrm{exchange } #1 \textrm{ with } #2}
\newcommand{\Nil}{\mbox{\bf nil}}
\newcommand{\In}{\mathsf{inOrder}}
\newcommand{\Post}{\mathsf{postOrder}}
\newcommand{\Pre}{\mathsf{preOrder}}
\newcommand{\Root}{\mathsf{root}}
\newcommand{\Parent}{\mathsf{parent}}
\newcommand{\Left}{\mathsf{left}}
\newcommand{\Right}{\mathsf{right}}
\newcommand{\Middle}{\mathsf{middle}}
\newcommand{\True}{\textbf{true}}
\newcommand{\False}{\textbf{false}}
\newcommand{\Print}{\mbox{\bf print }}
\newcommand{\ec}{({\bf Extra Credit})}
\newcommand{\note}{{\bf Note to Graders: }}
\newcommand{\Rotate}{\textsc{Rotate}}
\newcommand{\LRotate}{\textsc{LeftRotate}}
\newcommand{\RRotate}{\textsc{RightRotate}}

\newcommand{\notename}[2]{{\textcolor{red}{\footnotesize{\bf (#1:} {#2}{\bf ) }}}}
\newcommand{\sparsh}[1]{{\notename{Sparsh}{#1}}}

\begin{document}

\ifnum\me=0

% Collaborators (on a per task basis):
%
% Task 1: *********** INSERT COLLABORATORS HERE *********** 
% Task 2: *********** INSERT COLLABORATORS HERE *********** 
% etc.
%

\fi

% \handout{\name{} (\netid), Homework \num, Problem \arabic{pppp}}
% 	{\today}{Name: \name{} }{Due: \due}
% 	{Homework 2}

\newproblem{Collaborators}{0 points}
    \noindent
Add the names and NetID(s) of your collaborators at the start of your solution file. If you haven't collaborated with anyone then say none. Do not leave it blank! You are allowed to consult external resources but you must write the solutions on your own keeping your resources closed. You must mention your resources here. 
\ifnum\me<2
\begin{solution}\\

I have consulted lecture and recitation materials for this homework. Additionally, I have looked up some classical algorithms and their time complexities on the web with the explanations of why do we need them.

\end{solution}
\fi

\newproblem{Run-time Analysis}{\currentpoints}

For parts (a) to (c), state the worst-case time complexity and justify your answer. 
\begin{subtasks}

	\item \points{2} 

\begin{code}
    {\sc hungry}$(n)$\\
    \> $p = 0$ \\
    \>\For $i = 1$ \To $n$ \textbf{where} $i = i + 1$\\
    \> \> \For $j = i$ \To $n$ \textbf{where} $j = j + 1$\\
    \> \> \> \For $k = j$ \To $n$ \textbf{where} $k = k + 1$ \\
    \> \> \> \> $p = p + i$\\
    \> \Return p
\end{code}
	\ifnum\me<2
\begin{solution}\\

The worst time complexity for these three inner for loops in $\Theta(n^3)$. This is because each subsequent level runs $n$ times inside of each of the following ones, so we need to multiply $n \times n \times n$ to find the upper bound time complexity.\\

\end{solution}
	\fi

 \item \points{2}
\begin{code}
    {\sc famished}$(n)$\\
    \> $p = 0$ \\
    \>\For $i = 1$ \To $n$ \textbf{where} $i = i + 2$\\
    \> \> \For $j = 1$ \To $n$ \textbf{where} $j = j * 2$\\
    \> \> \> \For $k = 1$ \To $n$ \textbf{where} $k = k + 10$\\
    \> \> \> \> $p = p + i$\\
    \> \Return p
\end{code}

	\ifnum\me<2
\begin{solution}\\

Note that we always start from $1$ and run the loop until we reach $n$. However, what matters is the step size at which we approach the value $n$, in other words how quickly are we approaching it. The outer loop does it by approaching the value $n$ by adding 2 steps every iteration, the middle loop does so by multiplying the value by 2 at all times, and finally the inner one does so by adding 10.\\

Knowing this information, we can make some relevant conclusions. First of all, since we skip 2 steps every loop, the time complexity of the outer loop would be $O(n/2)$ and similarly for this inner-most loop, the time complexity would be $O(n/10)$. However the inner loop is a little tricky as we constantly divide the problem by 2, so the time complexity would be $O(log_2 (n))$. Since they are all within each other, we need to multiply them to get the tight bound, which is $\Theta(\frac{n}{2} \cdot log_2 (n) \cdot \frac{n}{10}) = \Theta(n \cdot log (n) \cdot n) = \Theta(n^2 \cdot log (n) )$, since we do not need to care about the constants.

\end{solution}
	\fi

\newpage

  \item \points{4} 
For simplicity, assume $n$ to be a power of $2$.
\begin{code}
    {\sc peckish}$(n)$\\
    \> \If $n==1$\\
    \> \> \Return 1\\
    \> $p = 0$ \\
    \>\For $i = n$ \To $1$ \textbf{where} $i = i - 56 $\\
    \> \> \For $j = 1$ \To $n$ \textbf{where} $j = j + 1$ \\
    \> \> \> \For $k = 1$ \To $100000000$ \textbf{where} $k = k + 0.1$\\
    \> \> \> \> $p = p + i$\\
    \> \Return $4*${\sc peckish}$(n/2)+p$
\end{code}

	\ifnum\me<2
\begin{solution}\\

This algorithm is called recursively on the input of size $n$ 4 times (dividing it by 2 every call), and adding a constant of $p$, which ends up being run $\frac{n}{56}$ times for the outer loop, $n$ more times for the inner loop, and a constant of $1,000,000,000$ times for the inner-most loop. Thus, the total time complexity would be described by the following recurrence relation: $T(n) = T(n/2) + n^2$ (where $n^2$ represents $p$). Using the expansion method one could get a time bound of $O(n^2)$.
%$\Theta(\frac{n}{56} \cdot n \cdot 1000000000 \cdot 4log_2 n) = \Theta(n^2 \cdot log_2 n)$%

\end{solution}
	\fi

\item \points{3}
Consider the following procedure that computes $x^n$ for given $x$ and $n$.
\begin{code}
    {\sc exponentiation}$(x,n)$\\
    \> $p = x$\\
    \> \For $i=2$ \To $n$ \textbf{where} $i = i + 1$\\
    \> \> $p = p*x$\\
    \> \Return $p$
\end{code}
Write a loop invariant for the algorithm and in a few lines argue that the algorithm is correct assuming that the loop invariant property is satisfied by the algorithm.
	\ifnum\me<2
\begin{solution}\\

Loop invariance in this case is that at the beginning of each iteration, the value of $p$ is actually $x^{i-1}$. It could be easily shown inductively. \\

Let's take i = 2, which is the base case. The value of $p = x$, which could also be represented in the following way: $p = x^{2-1} = x^{1} = x$. So the base case holds. \\

Similarly, loop holds for the $i-th$ iteration. When the loop terminates, $i=n+1$, the loop invariant still holds true. Then, $p$ will be the result of $x^{n+1 - 1} = x^{n}$. Thus, the algorithm is proved to be correct.\\


\end{solution}
	\fi

 \item \points{5}
 Consider the following procedure that inserts a key into a sorted array. Assume that the array has enough space to accommodate the new element (the key).
\begin{code}
    {\sc insert}$(x,n,A)$\\
    \> \For $i=n$ \To $1$ \textbf{where} $i=i-1$\\
    \> \> \If $A[i] > x$\\
    \> \> \> $A[i+1] = A[i]$\\
    \> \> \Else\\
    \> \> \> \textbf{break}\\
    \> $A[i+1] = x$\\
    \> \Return $A$
\end{code}

Write a loop invariant for the algorithm and prove its correctness using induction. Now argue very briefly that the algorithm is correct assuming that the loop invariant property is satisfied by the algorithm.
	\ifnum\me<2
\begin{solution} \\  

Step 1: Loop invariance: elements in the sub-array $A[i+1, ..., n]$ are always greater than $x$, and are \textbf{sorted}.\\

Step 2: Inductive proof:\\
At the first step, when $i = n$, we have empty array $A[i+1, ..., n]$, thus the condition holds.\\

There are two different cases we should consider for the general step. The first one is when we have the $A[i] > x$ then we need to "shift" all the elements to the right (keeping the relative \textbf{sorted order} of the elements). This happens until we find the case (2) where the $A[i]$ is equal or smaller than $x$. That's when we break out of the loop and assign our $x$ to it's assigned position. Note that the order of the elements in either case remains sorted and $x$ will eventually end up in the desired spot. Thus, general step holds true.\\

Finally, the loop is terminated whenever we have $i=1$, or $x$ being in the new spot in between of the element LSE to x on the left and strictly greater than on the right. The latter condition is already proven to be correct by utilizing the induction step above, and the first case of $x$ being inserted into the position 1 means that $x$ happened to be the smallest element in the entire array, thus it would be pasted into the first position while having all the rest of the elements shifted by one to the right. Thus, invariance holds true by induction proving the correctness of the algorithm. 


\end{solution}
	\fi
	
\end{subtasks}


\newproblem{Binary Search}
{\currentpoints}
\noindent

\emph{Luffy} just started as an intern at a \emph{Straw-Hat Shipping Company}. While wandering about, \emph{Luffy} messed around with a database, rotating a (sorted) row in a table by  $k$ elements. \emph{Luffy} in his infinite wisdom, decided to change the search module for that row instead of fixing his mistake.

\begin{subtasks}    
 
 \item \points{6}
 \emph Let $A[1 : n]$ be an array representing the row, but starting at entry $A[i]$
for some $i$, $1 \le i \le n$; i.e. $A[i] < A[i + 1] < \ldots  < A[n] < A[1] < \ldots  < A[i - 1]$. For example, if $A$ before rotating was $[1,2,3,4,5]$, after rotating by $2$ it becomes $[4,5,1,2,3]$.
 Write pseudocode to find the index of a given key in a sorted rotated array. The rotation value $k$ should \textbf{not} be an input to your procedure. Assume that the elements in the row are unique.
 
  \ifnum\me<2
\begin{solution}\\

Brief explanation of my course of actions. We know that the array is sorted at all times, the main problem here is to identify where to start and where to stop. We can use the modulo operator to figure out the rotating array.

Just to avoid confusion, I wrote an actual python code, thus the difference in indexing notation.

\begin{lstlisting}[language=Python]
def findIndexRotatedArray(Array, element):
  # perform a binary search to find an actual starting 
  #index in the rotating array (aka smallest element and it's index)
  start, end = 0, len(Array) - 1
  while start < end: 
      mid = (start + end) // 2
      if Array[mid] > Array[end]:
          start += 1
      else:
          end = mid
          
  # need to find the reference point to use for the
  # manipulations with the rotated array (if it is rotated)
  
  reference_point = start

  # perform binary search with accounts to what 
  # the reference pivot is
  start, end = 0, len(Array)
  
  while start < end:
      mid = (start + end) // 2
      middle_point = ((mid + reference_point - 1) % (len(Array) -1)) + 1
      
      if Array[middle_point] == element:
          return middle_point
      elif Array[middle_point] < element:
          start += 1
      else: 
          high = mid
  
  return None
\end{lstlisting}

\end{solution}
  \fi

   \item \points{3}
 State and justify the worst-case running time of the algorithm.
  \ifnum\me<2
\begin{solution}\\

The worst case complexity is $O(2log_2 n) = O(log n)$. This is because there is just a mathematical manipulation with finding the leverage point and utilizing it to get an actual "picture" of what the array before the rotation looked like.

\end{solution}
  \fi

  \item \ecpoints{3}
  Does your algorithm work with duplicates? If so, please explain why. If not, how would you modify your algorithm, and would that change the running time?
  \ifnum\me<2
\begin{solution}\\

HAVEN'T FINISHED YET!!

My algorithm would work with the duplicates only in certain cases and it would return the first occurrence of the element we are searching for. Additionally, it would be challenging to maintain this for all potential user-cases, because we are kind of binded to the reference point, which serves as a pivot. If we have duplicated values in the array, the values sometimes 

\end{solution}
  \fi
  

\end{subtasks}

 
\newproblem{Selection Sort}{\currentpoints}

 You are a teaching assistant for the course "Not-So-Basic Algorithms". After completing the course, you have tallied up the homework, quizzes, and exam scores for everyone. The final list is sorted alphabetically using first names but not according to the cumulative scores. The "TREBLA" interface expects the final list to be sorted according to the scores, and if there are students with equal scores, they should be alphabetically sorted. One of your students provides you with the following variant of selection sort to get the job done.

\begin{code}
		{\sc Selection-Sort}$(A,n)$\\
		\> \For $i=1$ \To $n-1$\\
		\> \> $max=1$\\
            \> \> $j = 2$ \\
            \> \> \While $(j \le n-i+1 )$\\
		\> \> \> \If $A[j]>A[max]$ \\
		\> \> \> \> $max=j$\\
		\> \> \> $j=j+1$ \\
		  \> \> swap$(A[j-1],A[max])$\\
    
	\end{code}

\begin{subtasks}

 \item \points{2} [$\textbf{Nami}(78), \textbf{Robin}(78), \textbf{Sanji}(82), \textbf{Usopp}(34), \textbf{Zoro}(34) $].\\
Trace the working of the algorithm with the above example input. For every iteration of the outer loop, clearly show the elements that will be swapped and the state of the array. Note that the input is initially alphabetically sorted.
	\ifnum\me<2
\begin{solution}\\

DIDNT SOLVE THIS ONE YET.

\end{solution}
	\fi
 

	\item \points{6} 
	Did the algorithm give you the desired result? (Were the students who scored the same also alphabetically sorted at the end?) If yes, then explain how the algorithm maintains stability. If not, modify the pseudocode to make it stable for all inputs and explain your modifications.
	\ifnum\me<2
\begin{solution}   INSERT YOUR SOLUTION HERE   \end{solution}
	\fi

\item \points{3}
	 Perform a worst-case runtime analysis on the original algorithm and, if applicable, on the second variant.
	\ifnum\me<2
\begin{solution}   INSERT YOUR SOLUTION HERE   \end{solution}
	\fi

\end{subtasks}

\newproblem{Merge Sort}{\currentpoints}

Recall Merge Sort, in which a list is sorted by first sorting the left and right halves, and then merging the
two lists. We define the 3-Merge Sort algorithm, in which the input list is split into 3 equal length parts (or
as equal as possible), each is sorted recursively, and then the three lists are merged to create a final sorted
list.


\begin{subtasks}
	\item \points{5} 
 Write pseudocode to implement the algorithm.
	\ifnum\me<2
\begin{solution}   
\begin{lstlisting}[language=Python, caption=Three-way Merge Sort Algorithm]

def merging(arr1, arr2): 
    i, j, k = 1, 1, 1
    a, b = len(arr1), len(arr2)
    c = a + b
    arr3 = [None * c]
    
    while i <= a and j <= b:
        if arr1[i] <= arr2[j]:
            arr3[k] = arr1[i] 
            i += 1
        else:
            arr3[k] = arr2[j] 
            j += 1
        
        k += 1
    
    while i <= a:
        arr3[k] = arr1[i] 
        i += 1
        k += 1

    while j <= b:
        arr3[k] = arr2[j] 
        j += 1 
        k += 1
    
    return arr3

def merge_sort(arr, l, r):
    # arg1: arr: Array of integers
    # arg2 and arg3: l and r, such that A[l, ..., r]
    if l < r: 
        q = (l+r) // 3
        merge_sort(arr, l, q)
        merge_sort(arr, q+1, 2q)
        merge_sort(arr, 2q+1, r)
        merging(arr1, arr2)

\end{lstlisting}

\end{solution}
	\fi

 \item \points{5} 
 State and justify the worst-case running time of your algorithm,
	\ifnum\me<2
\begin{solution}   INSERT YOUR SOLUTION HERE   \end{solution}
	\fi
	
	
\end{subtasks}





\newproblem{Applicative Questions}{\currentpoints}

\begin{subtasks}


\item \points{6} 

Write pseudocode for the following problem, given a set 
$S$ of $n$ integers and another integer 
$x$, determine whether S contains two elements that sum to exactly $x$. The algorithm should take $O(n\log{n})$ time in the worst case.
	
	\ifnum\me<2
\begin{solution}\\

Let's work on this from the perspective of sorting an array using MergeSort and then iteratively figuring out whether the array does have matching pair or not. Since MergeSort performs $O(nlog n)$ regardless of the input and the linear search is of linear time, I can conclude that the algorithm should take $O(n\log{n})$ time in the worst case.

\begin{lstlisting}[language=Python]

    twoSum(S, x):

        start = 1
        end = n
        # Assume the regular MergeSort Algo as mentioned
        # During the lectures 
        S = MergeSort(S)

        while(start < end):
        
            if S[start]+S[end]==target: 
                return True
                
            elif S[start]+S[end]>target:
                end -= 1
                
            else:
                start += 1
        
        return False

\end{lstlisting}


\end{solution}
	\fi

\end{subtasks}
\end{document}


