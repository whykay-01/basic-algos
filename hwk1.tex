% me=0 student solutions (ps file), me=1 - my solutions (sol file),
% me=2 - assignment (hw file)
\def\me{0} \def\num{1} %homework number

\def\due{11:55 pm on Thursday, February 1} %due date

\def\course{CSCI-UA 310-007, Basic Algorithms} %course name, changed only once

% **** INSERT YOUR NAME HERE ****
\def\name{Yan Konichshev}

% **** INSERT YOUR NETID HERE ****
\def\netid{yk2602}

% **** INSERT NETIDs OF YOUR COLLABORATORS HERE ****
\def\collabs{N/A, N/A}


\iffalse

INSTRUCTIONS: replace # by the homework number.  (if this is not
ps#.tex, use the right file name)

Clip out the ********* INSERT HERE ********* bits below and insert
appropriate LaTeX code.  There is a section below for student macros.
It is not recommended to change any other parts of the code.


\fi
%

\documentclass[11pt]{article}


% ==== Packages ====
\usepackage{amsfonts,amsmath}
\usepackage{latexsym}
\usepackage{etoolbox}
\usepackage{totcount}
\usepackage{fullpage}
\usepackage{graphicx}
\usepackage{tikz}
\usepackage{tikz-qtree}
\usepackage[bottom]{footmisc}
\usepackage{enumitem}
\usepackage{hyperref}
\usepackage{mathtools}
\DeclarePairedDelimiter\ceil{\lceil}{\rceil}
\DeclarePairedDelimiter\floor{\lfloor}{\rfloor}

\setlength{\footskip}{1in} \setlength{\textheight}{8.5in}

\newcommand{\handout}[5]{
	\renewcommand{\thepage}{#1, Page \arabic{page}}
	\noindent
	\begin{center}
		\framebox{ \vbox{ \hbox to 5.78in { {\bf \course} \hfill #2 }
				\vspace{4mm} \hbox to 5.78in { {\Large \hfill #5 \hfill} }
				\vspace{2mm} \hbox to 5.78in { {\it #3 \hfill #4} }
				\ifnum\me=0
				\vspace{2mm} \hbox to 5.78in { {\it Collaborators: \collabs
						\hfill} }
				\fi
		} }
	\end{center}
	\vspace*{4mm}
}

\newcounter{pppp}
\newcounter{pppc}
\newcommand{\prob}{\arabic{pppp}} %problem number
\newcommand{\increase}{\stepcounter{pppp}\stepcounter{pppc}} %problem number

% Arguments: Title, Number of Points
\newcommand{\newproblem}[2]{
	\increase
	\restartlist{subtasks}
	\ifnum\me=0
	\ifnum\prob>0 \newpage \fi
	\setcounter{page}{1}
	\handout{\name{} (\netid), Homework \num, Problem \arabic{pppp}}
	{\today}{Name: \name{} (\netid)}{Due: \due}
	{Solutions to Problem \prob\ of Homework \num\ (#2)}
	\else
	\section*{Problem \num-\prob~(#1) \hfill {#2}}
	\fi
}

\newlist{subtasks}{enumerate}{1}
\setlist[subtasks]{label={(\alph*)},resume}

\newcounter{numpppp}
\loop
\stepcounter{numpppp}
% workaround bug in totcount (means \newtotcounter{points\arabic{pointsct}})
\begingroup%
\edef\tempcounter@@name{points\arabic{numpppp}}%
\expandafter\newtotcounter\expandafter{\tempcounter@@name}%
\edef\tempcounter@@name{ecpoints\arabic{numpppp}}%
\expandafter\newtotcounter\expandafter{\tempcounter@@name}%
\endgroup
\ifnum \value{numpppp}<500 % max number of tasks supported
\repeat

% formating of output
\newcommand{\disppoints}[1]{%
	\texorpdfstring{(#1~\ifnumequal{#1}{1}{point}{points})}{}
}

% adds and displays
\newcommand{\points}[1]{%
	\texorpdfstring{\addtocounter{points\arabic{pppc}}{#1}\disppoints{#1}}{}%
}
\newcommand{\ecpoints}[1]{%
	\texorpdfstring{\addtocounter{ecpoints\arabic{pppc}}{#1}\ec \disppoints{#1}}{}%
}

% total points of current task         
\newcommand{\currentpoints}{% total points of current task   
	\texorpdfstring{\ifnumequal{\totvalue{ecpoints\arabic{pppc}}}{0}%
		{\total{points\arabic{pppc}} points}%
		{\total{points\arabic{pppc}}+\total{ecpoints\arabic{pppc}} points}}{}%
}


\def\squarebox#1{\hbox to #1{\hfill\vbox to #1{\vfill}}}
\def\qed{\hspace*{\fill}
	\vbox{\hrule\hbox{\vrule\squarebox{.667em}\vrule}\hrule}}
\newenvironment{solution}{\begin{trivlist}\item[]{\bf Solution:}}
	{\qed \end{trivlist}}
\newenvironment{solsketch}{\begin{trivlist}\item[]{\bf Solution
			Sketch:}} {\qed \end{trivlist}}
\newenvironment{code}{\begin{tabbing}
		12345\=12345\=12345\=12345\=12345\=12345\=12345\=12345\= \kill }
	{\end{tabbing}}


\newcommand{\hint}[1]{({\bf Hint}: {#1})}
% Put more macros here, as needed.
\newcommand{\room}{\medskip\ni}
\newcommand{\brak}[1]{\langle #1 \rangle}
\newcommand{\bit}[1]{\{0,1\}^{#1}}
\newcommand{\zo}{\{0,1\}}
\newcommand{\C}{{\cal C}}

\newcommand{\nin}{\not\in}
\newcommand{\set}[1]{\{#1\}}
\renewcommand{\ni}{\noindent}
\renewcommand{\gets}{\leftarrow}
\renewcommand{\to}{\rightarrow}
\newcommand{\assign}{:=}

\newcommand{\AND}{\wedge}
\newcommand{\OR}{\vee}

\newcommand{\For}{\mbox{\bf for }}
\newcommand{\To}{\mbox{\bf to }}
\newcommand{\DownTo}{\mbox{\bf downto }}
\newcommand{\Do}{\mbox{\bf do }}
\newcommand{\If}{\mbox{\bf if }}
\newcommand{\Then}{\mbox{\bf then }}
\newcommand{\Else}{\mbox{\bf else }}
\newcommand{\While}{\mbox{\bf while }}
\newcommand{\Repeat}{\mbox{\bf repeat }}
\newcommand{\Until}{\mbox{\bf until }}
\newcommand{\Return}{\mbox{\bf return }}
\newcommand{\Halt}{\mbox{\bf halt }}
\newcommand{\Swap}{\mbox{\bf swap }}
\newcommand{\Ex}[2]{\textrm{exchange } #1 \textrm{ with } #2}
\newcommand{\Nil}{\mbox{\bf nil}}
\newcommand{\In}{\mathsf{inOrder}}
\newcommand{\Post}{\mathsf{postOrder}}
\newcommand{\Pre}{\mathsf{preOrder}}
\newcommand{\Root}{\mathsf{root}}
\newcommand{\Parent}{\mathsf{parent}}
\newcommand{\Left}{\mathsf{left}}
\newcommand{\Right}{\mathsf{right}}
\newcommand{\Middle}{\mathsf{middle}}
\newcommand{\True}{\textbf{true}}
\newcommand{\False}{\textbf{false}}
\newcommand{\Print}{\mbox{\bf print }}
\newcommand{\ec}{({\bf Extra Credit})}
\newcommand{\note}{{\bf Note to Graders: }}
\newcommand{\Rotate}{\textsc{Rotate}}
\newcommand{\LRotate}{\textsc{LeftRotate}}
\newcommand{\RRotate}{\textsc{RightRotate}}
\begin{document}

\ifnum\me=0

% Collaborators (on a per task basis):
%
% Task 1: *********** INSERT COLLABORATORS HERE *********** 
% Task 2: *********** INSERT COLLABORATORS HERE *********** 
% etc.
%

\fi

% \handout{\name{} (\netid), Homework \num, Problem \arabic{pppp}}
% 	{\today}{Name: \name{} }{Due: \due}
% 	{Homework 1}



\newproblem{Collaborators}{0 points}
    \noindent
Add the names and NetID(s) of your collaborators at the start of your solution file. If you haven't collaborated with anyone then say none. Do not leave it blank! You are allowed to consult external resources but you must write the solutions on your own keeping your resources closed. You must mention your resources here. 
\ifnum\me<2
\begin{solution}   None   \end{solution}
\fi
 
\newproblem{Growth rate of Functions}{\currentpoints}


\begin{subtasks}

	\item \points{4}
	For each of the following functions $f(n)$, give a function $g(n)$ such that $f(n) = \Theta(g(n))$. For example,
	$5n + n^2 \log^{2} (n^2) = \Theta(n^2\log^{2} n)$.

	$$~~ n! + 34, ~~n^3 + n^2, ~~4^n + n^{3.5},~~~ n^{3.5} - 5n,~~5^{n}, ~~n^2\log(n) + 100n^2,~~\log(n) + 23  ,~~ n^n + \log(n)$$
	
	\ifnum\me<2
\begin{solution}   \\
\\
1) factorial grows faster than the constant
$$~~ f(n) = n! + 34, ~~ g(n) = n!; ~~ n! + 34 = \Theta(n!);$$
\\
2) cubic grows faster than the quadratic
$$~~ f(n) = n^3 + n^2, ~~ g(n) = n^3; ~~ n^3 + n^2 = \Theta(n^3)$$
\\
3) exponential grows faster than the polynomial
$$~~ f(n) = 4^n + n^{3.5}, ~~g(n) = ; ~~ 4^n + n^{3.5} = \Theta(4^n)$$
\\
4) polynomial to the power of 3.5 grows faster than the polynomial into the power of 1
$$~~ f(n) = n^{3.5} - 5n, ~~g(n) = n^{3.5}; ~~ n^{3.5} - 5n = \Theta(n^{3.5}) $$
\\
5) there is only one term, so the term itself is the tight bound
$$~~f(n) = 5^{n}, ~~g(n) = 5^{n}, ~~5^{n} = \Theta(5^{n})$$
\\
6) constants do not matter and $n^2\log(n)$ grows faster than a quadratic function
$$~~f(n) = n^2\log(n) + 100n^2, ~~g(n) = n^2\log(n); ~~n^2\log(n) + 100n^2 = \Theta(n^2\log(n))$$
\\
7) logarithmic function grows faster with comparison to the constant
$$~~f(n) = \log(n) + 23, ~~g(n) = \log(n); ~~\log(n) + 23 = \Theta(\log(n))$$
\\
8) $n^n$ is growing faster than the $\log(n)$
$$~~f(n) = n^n + \log(n), ~~g(n) = n^n; ~~n^n + \log(n) = \Theta(n^n)$$  
\end{solution}
	\fi

 \item \points{2}
	Sort all the resulting functions from part (a) in asymptotically increasing order of growth rate.

	\ifnum\me<2
\begin{solution} \\  
\\
$1) ~~ \log(n) + 23 $ \\
$2) ~~ n^3 + n^2 $ \\
$3) ~~ n^{3.5} - 5n $ \\
$4) ~~ n^2\log(n) + 100n^2 $\\
$5) ~~ 4^n + n^{3.5} $\\
$6) ~~ 5^{n} $\\
$7) ~~ n! + 34  $\\
$8) ~~ n^n + \log(n)$\\

$\log(n) \to n^3 \to n^{3.5} \to n^2\log(n) \to 4^n \to 5^{n} \to n! \to n^n  $


\end{solution}
	
	
	\fi
        
	\item \points{6}
	 Repeat the same exercise as part (a) for the following  $f(n)$ and give a function $g(n)$ such that $f(n) = \Theta(g(n))$. 
	$$~~n^4
	\log^5 n + n^5 \log^4 n,~~9^{n+\log_9 n} + n^29^n,~~3^{2n} + 4^n,
	~~\sqrt{n^{17}+12n^{16}},~~\log_7(n^{35}) + 10,~~\log_2^2 n + \log_4 n$$
	
	\ifnum\me<2
\begin{solution}\\
\\
1) poly-logarithmic is always slower than polynomial functions
$$~~ f(n) = n^4 \log^5 n + n^5 \log^4 n, ~~ g(n) = n^5 \log^4 n; ~~ n^4 \log^5 n + n^5 \log^4 n = \Theta(n^5 \log^4 n);$$
\\
2) after simplification, $n^2$ is faster than $n$. (Note that: $9^{n+\log_9 n} = 9^{n}n$)
$$~~ f(n) = 9^{n+\log_9 n} + n^29^n, ~~ g(n) = n^29^n; ~~9^{n+\log_9 n} + n^29^n = \Theta(n^29^n);$$
\\
3) Note that: $3^{2n} = 9^{n}$
$$~~ f(n) = 3^{2n} + 4^n, ~~ g(n) = 9^n; ~~3^{2n} + 4^n = \Theta(9^n);$$
\\
4) Leading term is $n^{17}$, so if we take $\sqrt{n^{17}} = n^{8.5}$ 
$$~~ f(n) = \sqrt{n^{17}+12n^{16}}; ~~ g(n) = n^{8.5}; ~~ \sqrt{n^{17}+12n^{16}} = \Theta(n^{8.5});$$
\\
5) We can forget about the constant and then take the power within the log out.
$$~~ f(n) = \log_7(n^{35}) + 10; ~~ g(n) = 35\log_7(n) = \log_7(n); ~~\log_7(n^{35}) + 10 = \Theta(\log_7(n));$$
\\
6) Poly-log functions grow faster than the regular logs.
$$~~ f(n) = \log_2^2 n + \log_4 n; ~~ g(n) = \log_2^2 n; ~~\log_2^2 n + \log_4 n = \Theta(\log_2^2 n);$$
\\

\end{solution}
	\fi


	\item \points{3}
	Sort all the resulting functions from part (c) in asymptotically decreasing order of growth rate.

	\ifnum\me<2
\begin{solution}\\
\\
From the fastest-growing function to the slowest-growing one.\\
\\
$n^29^n \to ~~9^n \to ~~n^5 \log^4 n \to ~~n^{8.5} \to ~~\log_2^2 n \to \log_7(n)$
\end{solution}
	
	
	\fi

\end{subtasks}

\newproblem{Asymptotic Comparison of Functions}{10 points}

\noindent
For the given pair of functions, express the asymptotic relation between them i.e., state whether f = $O$ / $o$ / $\Omega$ / $\omega$ / $\Theta$ (g). Multiple can apply! Justify your answer in a couple of lines.

\begin{subtasks}


 
 \item
 $f(n) = (n^{n})^{10}$; \quad $g(n)=n^{(n^{10})}$.
 
  \ifnum\me<2
\begin{solution}  \\

Since $f(n) = (n^{n})^{10}$ is equivalent to $g(n)=n^{(n^{10})}$, we can simply state that $f(n) = \Theta(g(n))$, which by definition implies that $f(n) = \Omega(g(n))$ \textbf{AND} $f(n) = O(g(n))$

\end{solution}
  \fi
  
\item  $f(n) = \log_4(n^5+5n^4)$; \quad
$g(n) = \log_{10}(n^{54})$.

\ifnum\me<2
\begin{solution}\\

Since they are both logs of the same power and even though their bases are different, we can conclude that they are both tight bounds of each other. That is: $f(n) = \Theta(g(n))$, which by definition implies that $f(n) = \Omega(g(n))$ \textbf{AND} $f(n) = O(g(n))$

\end{solution}
\fi
	
	
\item 
$f(n) = 4^{\log_2 n}$; \quad
      $g(n) = n^2 +  n (\log_4 n)^{10}$.

\ifnum\me<2
\begin{solution}\\

1) $f(n) = 4^{\log_2 n} = 2^{2{\log_2 n}} = 2^{\log_2 n^{2}} = n^2$\\

2) $g(n) = n^2 +  n (\log_4 n)^{10};$ or simply $O(n^2)$\\
\\
Thus, $f(n) = \Theta(g(n))$, which implies that it is also $f(n) = \Omega(g(n))$ \textbf{AND} $f(n) = O(g(n))$

\end{solution}
\fi	
	

\item $f(n) =n^{\log_2 20}$; \quad $g(n) = 20^{\log_2 n}$.

\ifnum\me<2
\begin{solution}   INSERT YOUR SOLUTION HERE   \end{solution}
\fi

\item $f(n) =\left(\frac{1}{2}\right)^n + 3 n^{6}$; \quad $g(n) = 2^{\log_2 n}$.

\ifnum\me<2
\begin{solution}\\

Since $\left(\frac{1}{2}\right)^n$ is converging to $0$ as $n$ is approaching infinity, we can conclude that $f(n) = O(n^6)$ because of its second (leading) term. Additionally, $g(n)$ could be expressed as an equivalent of a linear function $n$, since $2^{log_2 n}$ is exactly that.

Therefore, we can say that $g(n) = O(f(n))$.

\end{solution}
\fi
  



  
\end{subtasks}








\newproblem{Recurrence Relations}{\currentpoints}

 Consider the following recurrence relations, assume $T(1) = 1$ in both. 

\begin{subtasks}
	\item \points{5} 
 \begin{equation*}
     T(n) = 2 T(\floor*{\frac{n}{2}} + 7) ) + n
 \end{equation*}
    Solve using substitution method. (Hint: The class slides might give you a clue!)
	\ifnum\me<2
\begin{solution}\\

Let's start with a guess. My educated guess would be $T(n) \leq a*log n$ for all $n \geq n_0$ and both $a$ and $n_0$ are positive constants.



\end{solution}
	\fi
	
	\item \points{5}
 \begin{equation*}
     T(n) = 2T(n/16) + \log n
 \end{equation*}
	 Solve using expansion method. 
	\ifnum\me<2
\begin{solution}   

$T(n) = 2T(n/16) + \log n; ~~n > 1\\$
\\
$T(n) = 1; ~~n = 1\\$

$T(n/16) = 2T(n/256) + \log (n/16)$\\
\\
$T(n/16^{k}) = 2T(n/16^{k+1}) + \log n$

\end{solution}
	\fi
\end{subtasks}



\newproblem{Induction Method}{\currentpoints}

\begin{subtasks}

	\item \points{5} 
 \begin{equation*}
     f(n) = 5^n + 3^n
 \end{equation*}
	
 Show that the function is always divisible by 2.  
	\ifnum\me<2
\begin{solution}\\

Let us utilize the principles of mathematical induction to prove that the function $f(n)$ is always divisible by 2. 

\textbf{Base step:} 
Let's take as $n=0$. This will lead to $f(0) = 5^0 + 3^0 = 1+1 = 2$, which is divisible by 2.

\textbf{Inductive hypothesis:}
Let us assume $\exists k > 0$ such that $f(k) = 5^k + 3^k$, which would also produce a non-negative number $w$, s.t. $w ~~\% ~~2 == 0$.

\textbf{Inductive step:}
Let us take $k+1$ such that $f(k+1) = 5^{k+1} + 3^{k+1}$. Now, we want to show that $f(k+1) \% 2 == 0$, which shows the divisibility by 2.

\end{solution}
	\fi
	
	\item \points{5} 
	What is wrong with this \lq proof\rq?\\
 
   {\bf Theorem:} For every positive integer $n$, if $x$ and $y$ are positive integers with $\max(x,y) =n$ then $x=y$.\\

   {\bf Proof:}\\
   \textbf{Base step:} \\
   Suppose that $n=1$. If $\max(x,y) =1$ and  $x$ and $y$ are positive integers, we have $x=1$ and $y=1$.\\
   \textbf{Inductive step:} \\
   Let $k$ be a positive integer. Assume that whenever $\max(x,y) =k$ and  $x$ and $y$ are positive integers, then $x=y$. \\
   Now, let $\max(x,y) =k+1$, where  $x$ and $y$ are positive integers. Then  $\max(x-1,y-1) =k$, so by the inductive hypothesis, $x-1 = y-1$. It follows that $x=y$, completing the inductive step.
	
	\ifnum\me<2
\begin{solution}\\

First of all, the theorem is wrong. It is easily provable by contradiction. Take $x = 3, y = 5$. Note that both of them are positive integers. But the maximum of the two is 5. Thus, $n = 5$, but at the same time $x \neq y$.

Speaking about where the proof went wrong at the inductive step. If we are saying that $(x-1 = y-1) \rightarrow (x - 1 = y - 1 = k)$. Take $x or y = 1$, in this case $k = x - 1 = 1 - 1 = 0$, which is no longer a positive integer. 

\end{solution}
	\fi

\end{subtasks}

\newproblem{Applicative Questions}{\currentpoints}

\begin{subtasks}
	
	
	\item \points{5} 

Toe's Pizzeria is hosting a special event where they're generously offering a unique pizza adorned with a layer of supposedly edible gold dust to one fortunate (or perhaps not-so-fortunate) $NYU$ student. You find yourself as the recipient of this extraordinary pizza and decide to share the culinary joy in Washington Square Park. The pizza, $n$ units large, prompts you to extend the 
your generosity to your friend group, consisting of $8$ individuals, yourself included.

Embarking on a sharing spree, you slice the pizza into $8$ equal portions, devouring one while distributing the rest. Remarkably, $4$ of your friends belongs to other disjoint friend groups of $8$. Extending the generosity, these 4 friends repeat the slicing ritual. Amazingly, every friend group has 4 members who continue the ritual. This communal pizza enjoyment continues until the cut slice reaches a minimal size of $1$ unit. Formulate a recurrence relation in terms of the total number of slices and solve using any method.   
	\ifnum\me<2
\begin{solution}   INSERT YOUR SOLUTION HERE   \end{solution}
	\fi

\item (\textbf{Extra Credit}) \points{5} 

 \begin{code}
	\>{\sc Mystery}$(n)$:\\
        \> \> \If $n==1$ \Return $3$ \\
	\> \> \Else \Return $\mbox{\sc Arcane}(n) + \mbox{\sc Clandestine}(n)$
\end{code}

\begin{code}
	\>{\sc Clandestine}$(n)$:\\
	\> \> \If $n==1$ \Return $2$\\
	\> \> \Else \Return $\mbox{\sc Mystery}(n-1)$
\end{code}

\begin{code}
	\>{\sc Arcane}$(n)$:\\
	\> \> \If $n==1$ \Return $3$\\
	\> \> \Else \Return $\mbox{\sc Arcane}(n/2) + \mbox{\sc Arcane}(n/2)$
\end{code}



What mathematical function does the {\sc Mystery} procedure represent? Express it in terms of $n$, show your work while simplifying. (Hint: I wonder what mathematical function the {\sc Arcane} procedure represents!) 

	
	\ifnum\me<2
\begin{solution}\\   
{\sc Arcane} certainly represents a mathematical function that enables multiplication of an input by 3 $\to ~~f(n) = 3n$

{\sc Mystery}, on the other hand is a bit more complex, as it contains two different calls, which in turn produce more recursive calls.

\end{solution}
	\fi
 
\end{subtasks}








\end{document}


