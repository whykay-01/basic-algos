% me=0 student solutions (ps file), me=1 - my solutions (sol file),
% me=2 - assignment (hw file)
\def\me{0} \def\num{3} %homework number

\def\due{11:55 pm on Thursday, February 22} %due date

\def\course{CSCI-UA 310-007, Basic Algorithms} %course name, changed only once

% **** INSERT YOUR NAME HERE ****
\def\name{Yan Konichshev}

% **** INSERT YOUR NETID HERE ****
\def\netid{yk2602}

% **** INSERT NETIDs OF YOUR COLLABORATORS HERE ****
\def\collabs{N/A, N/A}


\iffalse

INSTRUCTIONS: replace # by the homework number.  (if this is not
ps#.tex, use the right file name)

Clip out the ********* INSERT HERE ********* bits below and insert
appropriate LaTeX code.  There is a section below for student macros.
It is not recommended to change any other parts of the code.


\fi
%

\documentclass[11pt]{article}


% ==== Packages ====
\usepackage{amsfonts,amsmath}
\usepackage{latexsym}
\usepackage{etoolbox}
\usepackage{totcount}
\usepackage{fullpage}
\usepackage{graphicx}
\usepackage{tikz}
\usepackage{tikz-qtree}
\usepackage[bottom]{footmisc}
\usepackage{enumitem}
\usepackage{hyperref}
\usepackage{mathtools}
\usepackage{graphicx}
\usepackage{listings}
\DeclarePairedDelimiter\ceil{\lceil}{\rceil}
\DeclarePairedDelimiter\floor{\lfloor}{\rfloor}

\setlength{\footskip}{1in} \setlength{\textheight}{8.5in}

\newcommand{\handout}[5]{
	\renewcommand{\thepage}{#1, Page \arabic{page}}
	\noindent
	\begin{center}
		\framebox{ \vbox{ \hbox to 5.78in { {\bf \course} \hfill #2 }
				\vspace{4mm} \hbox to 5.78in { {\Large \hfill #5 \hfill} }
				\vspace{2mm} \hbox to 5.78in { {\it #3 \hfill #4} }
				\ifnum\me=0
				\vspace{2mm} \hbox to 5.78in { {\it Collaborators: \collabs
						\hfill} }
				\fi
		} }
	\end{center}
	\vspace*{4mm}
}

\newcounter{pppp}
\newcounter{pppc}
\newcommand{\prob}{\arabic{pppp}} %problem number
\newcommand{\increase}{\stepcounter{pppp}\stepcounter{pppc}} %problem number

% Arguments: Title, Number of Points
\newcommand{\newproblem}[2]{
	\increase
	\restartlist{subtasks}
	\ifnum\me=0
	\ifnum\prob>0 \newpage \fi
	\setcounter{page}{1}
	\handout{\name{} (\netid), Homework \num, Problem \arabic{pppp}}
	{\today}{Name: \name{} (\netid)}{Due: \due}
	{Solutions to Problem \prob\ of Homework \num\ (#2)}
	\else
	\section*{Problem \num-\prob~(#1) \hfill {#2}}
	\fi
}

\newlist{subtasks}{enumerate}{1}
\setlist[subtasks]{label={(\alph*)},resume}

\newcounter{numpppp}
\loop
\stepcounter{numpppp}
% workaround bug in totcount (means \newtotcounter{points\arabic{pointsct}})
\begingroup%
\edef\tempcounter@@name{points\arabic{numpppp}}%
\expandafter\newtotcounter\expandafter{\tempcounter@@name}%
\edef\tempcounter@@name{ecpoints\arabic{numpppp}}%
\expandafter\newtotcounter\expandafter{\tempcounter@@name}%
\endgroup
\ifnum \value{numpppp}<500 % max number of tasks supported
\repeat

% formating of output
\newcommand{\disppoints}[1]{%
	\texorpdfstring{(#1~\ifnumequal{#1}{1}{point}{points})}{}
}

% adds and displays
\newcommand{\points}[1]{%
	\texorpdfstring{\addtocounter{points\arabic{pppc}}{#1}\disppoints{#1}}{}%
}
\newcommand{\ecpoints}[1]{%
	\texorpdfstring{\addtocounter{ecpoints\arabic{pppc}}{#1}\ec \disppoints{#1}}{}%
}

% total points of current task         
\newcommand{\currentpoints}{% total points of current task   
	\texorpdfstring{\ifnumequal{\totvalue{ecpoints\arabic{pppc}}}{0}%
		{\total{points\arabic{pppc}} points}%
		{\total{points\arabic{pppc}}+\total{ecpoints\arabic{pppc}} points}}{}%
}

\newcommand{\mixedpoints}[2]{%
	\texorpdfstring{%
		\addtocounter{points\arabic{pppc}}{#1}%
		\addtocounter{ecpoints\arabic{pppc}}{#2}%
		(#1 (+#2) points)
	}{}%
}


\def\squarebox#1{\hbox to #1{\hfill\vbox to #1{\vfill}}}
\def\qed{\hspace*{\fill}
	\vbox{\hrule\hbox{\vrule\squarebox{.667em}\vrule}\hrule}}
\newenvironment{solution}{\begin{trivlist}\item[]{\bf Solution:}}
	{\qed \end{trivlist}}
\newenvironment{solsketch}{\begin{trivlist}\item[]{\bf Solution
			Sketch:}} {\qed \end{trivlist}}
\newenvironment{code}{\begin{tabbing}
		12345\=12345\=12345\=12345\=12345\=12345\=12345\=12345\= \kill }
	{\end{tabbing}}


\newcommand{\hint}[1]{({\bf Hint}: {#1})}
% Put more macros here, as needed.
\newcommand{\room}{\medskip\ni}
\newcommand{\brak}[1]{\langle #1 \rangle}
\newcommand{\bit}[1]{\{0,1\}^{#1}}
\newcommand{\zo}{\{0,1\}}
\newcommand{\C}{{\cal C}}

\newcommand{\nin}{\not\in}
\newcommand{\set}[1]{\{#1\}}
\renewcommand{\ni}{\noindent}
\renewcommand{\gets}{\leftarrow}
\renewcommand{\to}{\rightarrow}
\newcommand{\assign}{:=}

\newcommand{\AND}{\wedge}
\newcommand{\OR}{\vee}
\newcommand{\For}{\mbox{\bf for }}
\newcommand{\To}{\mbox{\bf to }}
\newcommand{\DownTo}{\mbox{\bf downto }}
\newcommand{\Do}{\mbox{\bf do }}
\newcommand{\If}{\mbox{\bf if }}
\newcommand{\Then}{\mbox{\bf then }}
\newcommand{\Else}{\mbox{\bf else }}
\newcommand{\Elif}{\mbox{\bf elif }}
\newcommand{\While}{\mbox{\bf while }}
\newcommand{\Repeat}{\mbox{\bf repeat }}
\newcommand{\Until}{\mbox{\bf until }}
\newcommand{\Return}{\mbox{\bf return }}
\newcommand{\Halt}{\mbox{\bf halt }}
\newcommand{\Swap}{\mbox{\bf swap }}
\newcommand{\Ex}[2]{\textrm{exchange } #1 \textrm{ with } #2}
\newcommand{\Nil}{\mbox{\bf nil}}
\newcommand{\In}{\mathsf{inOrder}}
\newcommand{\Post}{\mathsf{postOrder}}
\newcommand{\Pre}{\mathsf{preOrder}}
\newcommand{\Root}{\mathsf{root}}
\newcommand{\Parent}{\mathsf{parent}}
\newcommand{\Left}{\mathsf{left}}
\newcommand{\Right}{\mathsf{right}}
\newcommand{\Middle}{\mathsf{middle}}
\newcommand{\True}{\textbf{true}}
\newcommand{\False}{\textbf{false}}
\newcommand{\Print}{\mbox{\bf print }}
\newcommand{\ec}{({\bf Extra Credit})}
\newcommand{\note}{{\bf Note to Graders: }}
\newcommand{\Rotate}{\textsc{Rotate}}
\newcommand{\LRotate}{\textsc{LeftRotate}}
\newcommand{\RRotate}{\textsc{RightRotate}}

\newcommand{\notename}[2]{{\textcolor{red}{\footnotesize{\bf (#1:} {#2}{\bf ) }}}}
\newcommand{\sparsh}[1]{{\notename{Sparsh}{#1}}}

\newcommand{\noname}[2]{{\textcolor{purple}{\footnotesize{\bf (#1:} {#2}{\bf ) }}}}
\newcommand{\tanmay}[1]{{\noname{Tanmay}{#1}}}

\begin{document}

\ifnum\me=0

% Collaborators (on a per task basis):
%
% Task 1: *********** INSERT COLLABORATORS HERE *********** 
% Task 2: *********** INSERT COLLABORATORS HERE *********** 
% etc.
%

\fi

% \handout{\name{} (\netid), Homework \num, Problem \arabic{pppp}}
% 	{\today}{Name: \name{} }{Due: \due}
% 	{Homework 3}

\newproblem{Collaborators}{0 points}
    \noindent
Add the names and NetID(s) of your collaborators at the start of your solution file. If you haven't collaborated with anyone then say none. Do not leave it blank! You are allowed to consult external resources but you must write the solutions on your own keeping your resources closed. You must mention your resources here. 
\ifnum\me<2
\begin{solution}   N/A   \end{solution}
\fi

\newproblem{Spiderman}{\currentpoints}


\noindent


The version of quicksort we saw in class does not optimize for duplicates. It includes duplicates of the pivot element in further recursive calls.
For this task, you are to develop a new partitioning procedure that accommodates duplicates. The idea is
to partition the array into elements less than the pivot, equal to the
pivot and greater than the pivot.

\begin{subtasks}

	\item \points{6} Use the idea to develop an in-place (uses constant space per recursive call) partitioning algorithm, providing pseudocode.
 
 \ifnum\me<2
\begin{solution}\\

\begin{lstlisting}[language=Python, caption=Partitioning for the three-way quicksort algorithm]

def partition(Array, pivot, low, high):
    # Algo that splits the array three-way for the anticipation of merging
    left = low
    right = low
    under = high

    while right <= under:
        if Array[right] < pivot:
            Array[left], Array[right] = Array[right], Array[left]
            left += 1
            right += 1
            
        elif Array[right] > pivot:
            Array[right], Array[under] = Array[under], Array[right]
            under -= 1
            
        else:  # case when we found a duplicate to a pivot
            right += 1

    return (left, right)


\end{lstlisting}


\end{solution}
	\fi

 \item \points{3}
        Utilize the partitioning algorithm to create a sorting algorithm. State and justify the worst-case running time.
  \ifnum\me<2
\begin{solution} \\

\begin{lstlisting}[language=Python, caption=Modified Quicksort algortithm]

def robust_quicksort(Array, low, high):

    if low < high:
        # randomly select the pivot
    
        pivot = Array[random.randint(0, len(Array - 1)]
        left, right = partition(Array, pivot, low, high) 

        # pay attention to the elements greater than the pivot
        quicksort(Array, low, left - 1)
        quicksort(Array, right, high)


\end{lstlisting}


\end{solution}
	\fi

  \item \points{2} 
Give an array that runs in time $\Theta(n^2)$ for the quicksort from class and $\Theta(n)$ for your algorithm. Why is that the case?
  \ifnum\me<2
\begin{solution} \\

Any array with duplicate values would work $\Theta(n)$ time. In some cases, even if we have at least a pair of duplicates we might end up with not having linear running time due to the risk of having a pivoting imbalance during the algorithm. It is the case, cause when we are choosing a pivot that is duplicated. In this case, the classic algorithm will end up with $n^2$, cause we basically turn it into the most basic compare and swap sorting algorithm.  

Here is the example: [5, 3, 2, 1, 0, 4] of an array which would follow the worst case scenario ($\Theta(n^2)$). On the other hand, arrays looking like this: [1, 1, 1, 1, 1], we will have a $\Theta(n)$

\end{solution}
	\fi

\end{subtasks}
 
\newproblem{Radix}{\currentpoints}

Consider an array $A[1, \ldots, t]$ containing $t$ numbers, where each number may have a different number of decimal digits. The total sum of all the digits of all elements is $n$. Additionally, the time required to find the number of digits of an element $A[k]$ is $O(i)$ where $i$ is the number of digits of the element. Given such an array, we wish to sort the numbers.

\begin{subtasks}

\item \points{3} 
 Think of a naive solution where we pad all the number with leading zeros to make all elements have the same number of digits as the the longest-digit number. We then run one iteration of {\sc RadixSort} on the resulting $t$
	numbers. Write the run-time of the naive algorithm as the function of
	$t,n$ and $L$, where $L\le n$ is the length of the longest integer in
	$A$. Specifically, what is the time of this naive algorithm when
	$t=n/2+1$, and one element has $n/2$ digits, while the remaining $n/2$
	elements have $1$ digit each. An example array of the worst case is $[1, 98765 , 4, 3, 5, 2]$.
	
 \ifnum\me<2
\begin{solution}\\
    Time complexity = $\Theta(n^2)$\\
    
    $\Theta(L*(t + 10)) = \Theta(t*L + 10L)$\\
    
    Note that $t = n/2 + 1$ and $L = n/2$, thus\\
    
    Time complexity = $\Theta((n/2 + 1) * (n/2) + 10 * (n/2)) = \Theta(n^2)$\\
\end{solution}
	\fi

 \item \points{6}
Give an algorithm that uses {\sc RadixSort} in a clever manner and works in worse-case time $\theta(n)$.
 \ifnum\me<2
\begin{solution}\\

The idea is in sorting the array by the number of digits into bins, where every single bin would be only the (the number of digits of an element A[k] is O(i)). And then performing radix sort in each individual category.

\begin{lstlisting}[language=Python, caption=Clever RadixSort]

def radix_sort_clever(Array):
    """
    L = max digit len of the array A
    n = total sum of all the digits of all elements
    bins = 2D array
    """
    bins = [1, ....., L]

    for number in A:
        # spread all the numbers in the respective bins
        num_digits = number_of_digits(number)
        bins[num_digits] = number

    result = []

    for bin in bins:
        sorted_bin = classic_radix_sort(bin)
        for i in range(len(sorted_bin)):
            result.append(sorted_bin[i])

    # this is going to be a sorted array of A
    return result
        

\end{lstlisting}

\end{solution}
	\fi

  \item \points{3}
State loop invariant(s) for your algorithm and assuming that they are true argue why your algorithm works.
 \ifnum\me<2
\begin{solution}\\

There are three different loop in-variants here. The first one is occurring at the stage 1, when we are grouping elements into the bins. The loop invariance at stage 1 is that an element is always ending up in the bin at index of the number of digits in that elements. And that the i-th index of the array bins will always contain elements with i number of digits.

The second phase where we are individually sorting each bin using the $classic_radix_sort$ algorithm has its own loop invariance. The loop invariant here is simply that at any given loop i we have the i-th digit of that very bin sorted as well as all the previous digits before the i-th digit are sorted as well. 

The last invariance is coming at the inner loop, where we are appending all the values in bins to the final output result array. The invariance here is simply that all the numbers we append to the result array are already sorted in ascending order, so we append it to the final array with the result.

\end{solution}
	\fi


 \item \points{3}
\textbf{Properly} justify the running time of your algorithm.
 \ifnum\me<2
\begin{solution}\\

We are using $n$ to represent the total sum of all the digits of all elements in the array A. Thus, we would be computing the total n in the first loop, which is happening because we go over the number and calculate the number of digits in it. Thus, we would end up having O(n) time for the first loop.

Radix sort that we would have runs in O(n) as well, cause we would have at most L number of bins in the $bins$ 2D array. 

And the very last loop is essentially the anti-action of what we did in the first loop, which also results in O(n) time, cause we go over the numbers in sorted bins and simply append them all.

Thus, final complexity would be $3\cdot O(n) = O(n)$

\end{solution}
	\fi
 
	


\end{subtasks}

\newproblem{The $k$ Largest Elements}{\currentpoints}


Consider a data structure $D$ that maintains the $k$ ($k \leq n$) largest elements of an array $A = A[1,\ldots,n]$. We can implement $D$ using one of the following data structures: 

\begin{enumerate}[label=\Roman*.]   
\item  Unsorted array of size $k$
\item  A sorted array of size $k$
\item  Min-heap of size $k$
\end{enumerate}


We want the data structure to be able to facilitate

\begin{itemize}
	
	\item {\sc Initialize}($A,n,k$) that initializes $D$ for a
	given array $A$ of $n$ elements.
	
	\item {\sc Largest}($D$), that returns the $k$ largest elements of
	$A$ {\em in sorted order}.
	
	\item {\sc Update}($D, x$), that updates $D$ when an element $x$ is
	appended to the array $A$.
	
\end{itemize}

\textbf{Note:} We do \textbf{NOT} want psuedocode for this question! You have a describe your algorithm stepwise and explain your steps.

\begin{subtasks}
	
	\item \mixedpoints{6}{2}
For all the data structures, show how the {\sc Initialize} procedure can be implemented to run in  \textbf{expected} time $O(n+k\log n)$. For extra credit, give an implementation that runs in $O(n+k\log n)$ time for the \textbf{worst case}.
 
	\ifnum\me<2
\begin{solution}
\begin{enumerate}[label=\Roman*.]   
\item  \textbf{Unsorted array of size} $k$\\

Find the $k-th$ largest element using the algorithm covered during the lecture, called QuickSelect, which allows us to pick a $i-th$ rank number of the array in linear $O(n)$ time. Additionally, it is worth mentioning that all the elements that are greater than $k-th$ element would be on the right-hand side of that very element. After such pivot element is found, we could partition around that element and copy paste all the k largest elements into the newly initialized array D, which would take $O(k)$ running time. Final time complexity would be $O(n+k)$.\\

\item  \textbf{A sorted array of size} $k$\\

We need to perform the $k-th$ largest element using the QuickSelect algorithm, which allows us to pick a $i-th$ rank number of the array in linear $O(n)$ time. After that we want to copy all these elements that are greater than the k-th largest element into the D, which once again takes $O(k)$ time. After that we need to sort D array using MergeSort, which would take $O(k\cdot logk)$. Thus, final time complexity would be $O(klog k + n)$ time. \\


\item  \textbf{Min-heap of size} $k$\\

First, create a minheap of the k elements of array A, which would take $O(k)$ time. Go over the rest of the elements in the array starting with the $(k+1)^th$ element and compare them with the root element of the min heap. There are two scenarios we develop in case we encounter the element.

1. If element is larger than the root of the heap then remove the root element and propagate it down the tree, which would take $O(log k)$, where $log k$ is the height of the tree. 

2. If the element is smaller than the root, we don't care about it as we are interested in the largest elements only.

Finally, we are performing this loop for n-k times, and thus the worst time complexity would be $O((n-k) * log k)$ because we have to assume that in worst case we would have to shift the root of the heap all the time. That is simply upper-bounded by $O(nlogk)$, which is the final complexity.

\end{enumerate}

\end{solution}
	\fi
	
	\item \points{6} 
 For all the data structures, state and show how the {\sc Largest} procedure can be implemented to run in best time you can think of.
	
	\ifnum\me<2
\begin{solution}   
\begin{enumerate}[label=\Roman*.]
\item  \textbf{Unsorted array of size} $k$\\

We have to sort the array first, which would take at best $O(k log k)$ for 

\item  \textbf{A sorted array of size} $k$\\
\item  \textbf{Min-heap of size} $k$\\
\end{enumerate}

\end{solution}
	\fi
	
	\item \points{6}
 For all the data structures, state and show how the {\sc Update} procedure can be implemented to run in best time you can think of.
	
	\ifnum\me<2
\begin{solution}   
\begin{enumerate}[label=\Roman*.]
\item  \textbf{Unsorted array of size} $k$\\



\item  \textbf{A sorted array of size} $k$\\
\item  \textbf{Min-heap of size} $k$\\
\end{enumerate}
\end{solution}
	\fi
 
\end{subtasks}

\newproblem{Applicative Questions}{\currentpoints}

\begin{subtasks}

\item \points{5}
Assume that we are given $n$ bolts and $n$ nuts of different sizes, where each bolt exactly matches one nut. Our goal is to find
the matching nut for each bolt. The nuts and bolts are too similar to compare directly; however,
we can test whether any nut is too big, too small, or the same size as any bolt. Prove that in the worst case, $\Omega(n \log n)$ nut-bolt tests are required to correctly match up the nuts and bolts.
\ifnum\me<2
\begin{solution}\\  
We are going to be using the decision tree to analyze and prove that the time complexity of the algorithm is $\Omega(nlogn)$. 

The decision tree itself has internal nodes and leaves, where internal nodes would be representing the comparison between the nuts and bolts and leaf nodes would be representing the match between a nut and a bolt. Since one nut can correspond only to one bolt, our leaf nodes need to show that mapping once it occurs. 

Since we have n nuts and n bolts, there is n! potential ways we can match nuts and bolts with each other. Therefore, to represent all the scenarios we would need to have n! leaves for all potential scenarios. Each pair states whether the nuts $><=$ to the size of the bolt. 

According to the tree theory, the number of all the nodes in the tree can be at most $2^t$, where t represents the height of the tree. We end up with the following inequality about the nut-bolt tests. $n! \leq 2^h = h \geq log(n!)$. This could be reasonably solved using the approximation of the factorial. $n! \approx nlog_2(n) - nlog_2(e) + O(log_2 n)$. Thus, we can see that the $nlog_2(n)$ is a fastest growing term, thus the height of the tree is having a lower bound of $\Omega(n log n)$ as long as $h \geq log_2(n!)$

\end{solution}
	\fi

 \item \points{5}
Let $S = \{e_1, e_2, \ldots, e_n\}$ be a set of $n$ distinct integers in a range large enough to mean that radix sort does not run in linear time. We aim to give an expected $O(n)$ time algorithm to report all pairs $(e, f)$ of integers in $S$ such that $e = 5f$.
\ifnum\me<2
\begin{solution}   \\

The effective solution like this would involve using hash maps. We simply create the hash table with the e-values from S, where e values would serve as the keys of the hash table. For each of the e we would need to add f = 5/e, and add it as a value for the each respective key. After that, iterate over the \textbf{values} and perform constant time lookups whether the values are present in the keys. If they are report a pair e and its respective f. The entire complexity of the algorithm would be $O(n)$, since we just simply linearly go over all the values of the set S and computing the corresponding f values, after which we go over that again and look for matches.

\end{solution}
	\fi



\item \ecpoints{5} 
 Suppose you are given an array $A[1 : n]$ of integers in the range $[1, m]$, where $m > n$. Additionally an uninitialized array $B[1:m]$ (don't assume anything about the values in B initially) is also available to you. 

 Show how to identify all duplicate
items in A in wosrt-case time $O(n)$ (the first instance of an item in increasing i index order is a nonduplicate, the others are duplicates). You can report a duplicate item as the pair $(A[i], i)$. You will need to use array B to help.



	\ifnum\me<2
\begin{solution}   
This is very similar to just using the hash maps, but in this case we will just use the secondary array B to keep track of all the values in A. \\

Here is an example code of how this could work: \\

\begin{lstlisting}[language=Python, caption=Clever RadixSort]
def find_duplicates(A, n, m):
    """
    A = [1, 4, 3, 2, 0]
    B = [0, 0, 0, 0, 0, 0]
         ^  ^  ^  ^  ^  ^
         0  1  2  3  4  5
    """
    # Initialize array B with zeros
    B = [0] * m

    duplicates = []

    # go over every element in array A
    for i in range(n):
        # element has been encountered before?
        if B[A[i]] == 0:
            # mark a element for the first time
            B[A[i]] = 1
        else:
            # has been encountered before --> a duplicate
            duplicates.append((A[i], i))

    return duplicates
\end{lstlisting}


\end{solution}
	\fi


\end{subtasks}
\end{document}


